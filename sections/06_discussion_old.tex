
\section{Discussion} \label{sec:discussion}

The results highlight the advantages of using a full-state feedback controller that compensates also for estimated disturbance torques and for viscous torque losses of the motor. The major benefit on the use of the full state feedback control is high exhibited transparency during the free motion task. This means that the exoskeleton can affect less the user desired motion and at the same time the robot is able to more accurately identify the user intention (the human forces/torques).
\par Although the basic state feedback control (JTFC2) presents an average force modulus at the end-effector similar to the passivity-based feedback control (JTFC3), it hinders the user voluntary motion more than the other controls do. In fact the end-effector trajectory due to the control JTFC2 is the farthest from the desired one. This is because the JTFC1 and the JTFC3 take into account (although different ways) both the link's inertia and the motor's inertia, whereas the JTFC2 control considers only the motor's inertia. The full state controller explicitly estimates the load torques and feed-forwards this contribution by multiplying it by a gain. On the other side, the passivity-based controller imposes by control the desired link's inertia, thus this is a direct objective of the control.
\par The high transparency (the average force modulus at the end-effector is less than 6 N for JTFC1) is also due to the effect of the dynamic compensations. A correct estimation of the joint acceleration is crucial to obtain a transparency enhancement, this is the reason why the dynamic contributions are weighted by a constant less than 1. In fact only with a perfect acceleration estimation it can be obtained a high transparency keeping a stable behavior. The proposed methodology for the estimation of the acceleration through torque sensor's data and motor's data can help to improve the acceleration calculation.
\par An important result is the wide range of stable impedances that the system can render. The Rehab-Exos was able to render a flat surface with a stiffness equal to 40 kN/m with all the three compared control laws with different performances but still preserving stability. This is certainly due to inherent mechanical damping of the system.
\par The mechanical design of the exoskeleton influences its performances. The residual torques at the joint are basically the effects of the unmodeled link inertia and the joint friction. A more light design made by small motors and small transmission ratio will lead to a more backdrivable solution with the price of a less torque available at the joint. This could be a solution to experience an even more transparent device. Moreover the torque sensor requires a more robust design; more in detail, to obtain a smaller sensitivity to non-axial loads a spoke with wider beams can be implemented.
\par The choice of a joint with an active impedance by control based on a torque sensor presents a valid alternative to the passive inherent compliant actuators in order to achieve a more compact and simpler mechanics and electronics. The proposed torque control combined with the joint mechanics allow to build safe and responsive control strategies suitable for rehabilitation and assistance. 
