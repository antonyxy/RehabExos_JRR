\section{Conclusion} \label{sec:conclusion}
This paper presents the Rehab-Exos exoskeleton design and in particular the design of the joint torque sensors based on strain gauges. Some sensor's issue have been explained and two possible hypotheses have been proposed. Then an interaction torque control has been developed and validated by experimental tests such as the transparency test and the haptic interaction tasks. The kinematics and dynamics of the device are calculated by a full dynamics model implemented in a centralized torque control. The torque tracking for each joint is performed by single-joint full-state Kalman filter and a torque feedback controller. The centralized control provides to each single-joint observer the desired torque for force feedback and an estimation of the joint torques due to links dynamic loads to be compensated by the control as feed-forward contributions. The developed full-state feedback control was then compared with a basic feedback control and a passivity-based feedback control. Results show how the  full-state approach is effective for estimating the human interaction force cleaned up of the inertial and gravity contributions due to the non negligible mechanical properties of the exoskeleton structure. The full-state feedback control is more accurate and transparent than the other two controls. The proposed control strategy combined with the presence of a joint torque sensor can enhance the performances of the human-robot interaction based on exoskeleton even in presence of non backdrivability.
%
