% As a general rule, do not put math, special symbols or citations
% in the abstract or keywords.
% in the abstract or keywords.
\begin{abstract}
	In the last two decades, the research on exoskeletons has been increasing for several applications. The mechanical architecture of the exoskeleton directly influences its performances in term of torque and position control. An other aspect is the complexity of the joint design and the number of the required actuators and sensors per joint. In this work we present an upper-limb exoskeleton designed with compact elastic joints with torque sensors based on strain gauges. The torque sensor performances and the design aspects that affect the unwanted non-axial moment load crosstalk are addressed in this study. The joints have a high reduction ratio and they display an output torque of 150 Nm, nevertheless, they are transparent by control and show a low error in both haptic force rendering and position control. Good transparency and force rendering are obtained by model-based full-state feedback control. The control schema takes into account part of the torque sensor non-linearities. Performances have been evaluated using different controls showing that this solution is a valid trade-off among exoskeleton complexity, maximum torque, transparency and haptic capabilities.
\end{abstract}