% Point one description 


% Point one description 
\begin{point}
While many of the techniques are standard, what is commendable is that
the authors implemented this control architecture in real hardware. The
importance of this should not be understated. This represents a
significant amount of work. With some revisions to enhance the clarity
of the presentation, This could be an impactful paper.
\end{point}

% Our reply
\begin{reply}
	We thank the reviewer for this comment, we agree and we confirm that demonstrating in real hardware requires bigger effort that allows to evaluate experimentally some theoretical assumptions.
	
\end{reply}


\begin{point}

The biggest issue I have with the paper is that the reader has to get
to the end of the paper to discover what the authors' central
contribution is. The thesis of the paper, found in the conclusion, is
that it provides evidence that with a suitable control architecture, it
is not necessary for a rehabilitation upper limb exoskeleton to be
backdrivable by the user. This is important. However, from the start of
the paper, it is not clear where the paper is going. The introduction
is overly long, and cites work only loosely related to the work
presented, e.g. works on lower limb exoskeletons. It then launches into
the (previously published) description of the hardware, and the
narrative as written gives the impression that this is a paper about a
new exoskeleton, when in fact the novelty is the control architecture
for the exoskeleton, particularly a comparative hardware evaluation of
several alternatives. This should be made clear from the outset, even
in the abstract. The abstract should list the types of controllers
evaluated and which was found to be superior in the given context.
\end{point}


\begin{reply}
We followed this important suggestion, that allowed us better focus on the paper.

The introduction was partly rewritten and both abstract and introduction were better focused on the results of the paper.
We agree that were some references that were loosely connected to the actual work, so we shortened some parts were detailed description of other exoskeletons implementation was provided.
\end{reply}


\begin{point}
Throughout the paper, there are some odd choices that were made in the
work. For example, to combat misalignment of the strain gauge on the
spring element of the torque sensor, the authors use a neural network
with training data. There are well-established methods to address this,
such as differential pairs and using a hardware calibration. Why the
neural network? It seems to cost more in the way of development for
something that is less reliable.
\end{point}

\begin{reply}

The neural network is an adaptive modeling approach that without prior knowledge of internal parameters is capable of reaching with only one session of training a good performance. Although we agree with the reviewer that there are more specific methods available in literature, we considered that the ANN approach was more flexible thanks to its adaptivity and due to the complexity of the model we consider that this was an acceptable approximation.  
We added in the text that the clarification that  ANN method was chosen due to its adaptive nature.
\end{reply}



\begin{point}



 The two-mass model in Fig. 11 is used
to approximate the linkages of the robot, but this neglects the dynamic
equations of this links. Some justification might be found for this in
the fact that the gear ratio is on the order of 100, but the authors
make no such justification, then they mention joint torques due to
centripetal and coriolis terms, which would suggest they are including
these terms, but the narrative is ambiguous.
\end{point}



\begin{reply}
We have tried to improve the description of dynamic equations to make more clear all the assumption behind the presentation of our dynamic model.
The link dynamics is modeled completely, taking into account in matrices M and C both inertias and Coriolis effects. In section III.B (equation 10) and in appendix I there is a detailed explanation on how then the dynamics is split into different contributions.
 Of course in the two-mass model of section III.A for each link all the inertias are reduced to the axis of the link, so that only the dynamics acting on the motor axis is considered. 
 The joint coupling is considered in the multi-joint dynamic model in section III.B that is one of the contributions of the paper.
 In equation (9) we have now added some braces that highlight the contribution of both joint and motor dynamics. The compensation of dynamics was verified by the good results obtained in the experimental characterization of transparency.
\end{reply}




\begin{point}

 They speak of a natural
frequency in the discussion around Fig. 12, which suggests a
linearization was performed (where the resonance would shift with
operating point). Fig. 12 shows the magnitude plot but not the phase
plot. The phase would be illustrative of how well the linearization
captures the behavior of the system
\end{point}
\begin{reply}
We agree with the reviewer that the phase plot would be informative, so we have added now the phase plot in the second part of the same figure.
\end{reply}



\begin{point}

 The estimation architecture in
(13) appears to be a predictor architecture only, yet (17) suggests a
sort of Leuenberger Observer architecture (although they call it a
Kalman Filter) No indication of how the observer gains were calculated
or covariance matrices used is given. I am also not clear on why they
needed to observe the joint accelerations to implement the controller.

\end{point}
\begin{reply}

 We have tried to be cleared by adding the following text
 {\em The gain $L$ was found resolving the problem of a Kalman optimum observer based on the experimental covariance data of measurement and process noise. Measurement noise was derived from motor encoder ${\theta}_{m,i}$ measurements, that mainly takes into account the encoder quantization and motor acceleration $\ddot{\theta}_{m,i}$ estimation through (13), thanks to the available torque measurement.}

Actually this is not a reduced observer, but an additional state was added to estimate the motor dynamics. The additional term is used for compensation of motor dynamics.
\end{reply}



\begin{point}
	
	Typically in robotic systems (and others driven by Newtonian Mechanics)
	the acceleration is not a state of the system (unless it happens to
	emerge in place of other positions and velocities by some state
	transformation). The rationale for observing the second derivative is
	not clear. Equation (17) and the surrounding discussion does not help
	the reader evaluate what physical quantities correspond to y (other
	than looking at the products of the state), and in the block diagram on
	the next page it seems that torque is what is measured, so it
	ambiguous.
	
\end{point}
\begin{reply}
A good estimation of the acceleration is crucial to provide dynamics compensation.
More in detail, the estimated joint accelerations have been multiplied by $\hat{M}_i$ that is the inertial term of the i-th joint (the reviewer can find a graphical representation in the new figure 9, at the top-left corner). 
Regarding the block diagram of the old figure 14, the inputs are position and acceleration. To improve clarity, the old figures 13 and 14 has been merged.
\end{reply}




\begin{point}
	It was not clear why in the experiment they had the user
	grab the end effector of the exoskeleton rather than wearing it. If the
	user is wearing the exoskeleton, there would likely be forced applied
	to the various links along their length, which would "break" the
	traditional formulation that depends on computing joint torques through
	the Jacobian transpose, which may call into question some of the
	formulation presented.
\end{point}
\begin{reply}
The user actual wear the exoskeleton, that has two contact points, at arm and level of hand. The dynamic system will lead to compensation of external forces based on the actual readings that are made at level of joint torque sensing.  In equation (23) the effect of additional forces introduced at single points that are not at the end-effector is canceled as an effect of disturbance estimation made by $\tau_{d}$.
In haptic rendering, we expect that forces are rendered at the end-effector.
\end{reply}


\begin{point}
	In section IV, the three controller types are listed, but it would be
	helpful to say, "We evaluated three different controllers". Until the
	end of the narrative, I was under the false impression that the other
	two controllers were embellishments of the first, whereas they were
	three completely disparate solutions.
\end{point}
\begin{reply}
The reviewer is correct, the other 2 controllers have to be considered benchmark architectures. This has been not pointed out since the abstract and in the introduction, so that the reading flow should be significantly improved-
\end{reply}


\begin{point}
	
	By and large the figures are well done. I might suggest that the
	authors display Fig. 21 in three-dimensional space rather than three
	separate curves superimposed on the same plot for each coordinate. 
	
\end{point}
\begin{reply}
This figure has been completely revised, and the plots have been split in different panels to take into account the reviewer suggestion.
\end{reply}


\begin{point}
	
	
	The authors also have several English mistakes that appear throughout,
	for example, using "performances" instead of "performance", and often
	use the wrong preposition in idomatic expressions, e.g. "inspired to"
	rather than "inspired by". There are also several long wordy
	expressions, e.g. "due to the absence of elastic or damping elements",
	which should be expressed more concisely in English, e.g. "which have
	no elastic or damping elements". These are minor, but should be
	resolved in a revision as they disrupt the reader.
\end{point}
\begin{reply}
We thank you the reviewer for pointing out these mistakes, they have been replaced in all occurrences
\end{reply}





\subsection*{Minor}

%% Use the short-hand macros for one-liners.
%\shortpoint{ Typo in line xy. }
%\shortreply{ Fixed.}

\begin{point}
%The authors present a design of a 5-joint, 4 degree of actuation upper
%limb exoskeleton for rehabilitation purposes, with a high gear ratio
%transmission. This makes the exoskeleton non-backdrivable. The main
%contribution is the evaluation of three different control architectures
%for the joints. The authors compare a Full State Feedback Controller, a
%`basic' state feedback controller (a simple proportional-derivative
%task space controller), and a passivity based controller. A summary of
%the hardware design is given, along with an observer architecture to
%estimate the derivatives of the angle from encoder measurements.






Ultimately I believe that a new draft that is tightly focused and
guides the reader to understand the key elements of the work conducted
and its significance throughout the read, rather than finally
understanding during the discussion section, it will be a good paper. \end{point}

% Our reply
\begin{reply}
We thank you very much the reviewer for all the helpful comments. We have now stressed more in the introduction and the paper all the key elements and findings, hoping that will improve paper readability.
 
\end{reply}



%\subsection*{Minor}
%
%% Use the short-hand macros for one-liners.
%\shortpoint{ Typo in line xy. }
%\shortreply{ Fixed.}